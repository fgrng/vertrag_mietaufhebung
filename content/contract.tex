% =====================================================================
% =====================================================================
% ---
% ---

% =====================================================================

\subsubsection*{Beendigung des Mietverhältnisses}

Das zwischen den Vertragsparteien bestehende Mietverhältnis über das Zimmer
zu Wohnzwecken in der o.g. Wohnung wird in beiderseitigem Einvernehmen zum
\textbf{\CONTRACTenddate{}} beendet.

\subsubsection*{Räumung und Übergabe}

Die Untermieter*in verpflichtet sich, die o.g. Wohnung spätestens bis zu
dem im Abschnitt \emph{Beendigung des Mietverhältnissen} genannten
Beendigungstermin zu räumen und der Hauptmieter*in einschließlich aller
Wohnungs- und Haustürschlüssel zu übergeben.

Kommt die Untermieter*in mit der Rückgabe der Mietsache in Verzug, hat sie
für jeden angebrochenen Kalendertag der Verspätung eine
Nutzungsentschädigung in Höhe von \textbf{\CONTRACTdelay{} EUR} an die
Hauptmieter*in zu zahlen. Die Untermieter*in verpflichtet sich diesen
Betrag innerhalb von zwei Wochen nach Beendigung des Mietverhältnisses auf
das folgende Konto zu überweisen.

\vspace{2ex}
\begin{addmargin}[.25\textwidth]{.25\textwidth}
  Inhaber*in \MYdotfill \textbf{\ACCOUNTowner} \\
  IBAN \MYdotfill \textbf{\ACCOUNTiban} \\
  BIC \MYdotfill \textbf{\ACCOUNTbic} \\
  Bank \MYdotfill \textbf{\ACCOUNTinstitute}
  Betreff \MYdotfill \textbf{\ACCOUNTinstitute}
\end{addmargin}
\vspace{2ex}

Setzt die Untermieter*in den Gebrauch der Mietsache nach der Beendigung des
Mietverhältnisses fort, gilt das Mietverhältnis nicht als stillschweigend
fortgesetzt; §545 BGB findet keine Anwendung.

\subsubsection*{Schönheitsreparaturen}

Die Vertragsparteien sind sich darüber einig, dass die Untermieter*in auf
Grund Abschnitt \emph{Bezugnahme auf den Hauptmietvertrag} des zwischen ihr
und der Hauptmieter*in geschlossenen Mietvertrages gemäß \CONTRACTaesthetic
des Hauptmietvertrags im Zeitpunkt der Beendigung dieses Mietverhältnisses
zur Durchführung der Schönheitsreparaturen verpflichtet ist.

Die Untermieter*in verpflichtet sich, die oben genannten
Schönheitsreparaturen bis zur Beendigung des Mietverhältnisses fachgerecht
durchzuführen.

\subsubsection*{Mietkaution}

Die Hauptmieter*in verpflichtet sich, die von der Untermieter*in gezahlte
Kaution in Höhe von \textbf{\CONTRACTdeposit{} EUR} einschließlich Zinsen
spätestens bis zum Ablauf von 12 Monaten seit der Beendigung des
Mietvertrages an dem Mieter zurückzuzahlen. Diese Pflicht besteht nicht,
wenn und soweit der Hauptmieter*in fällige Gegenansprüche aus dem
Mietverhältnis zustehen, mit denen die Hauptmiter*in die Aufrechnung
erklärt hat.

\subsubsection*{Betriebskosten}

Die Hauptmieter*in verpflichtet sich, über die Betriebskosten
schnellstmöglich, spätestens jedoch bis zum \textbf{\CONTRACTutilities}
abzurechnen. Beide Vertragsparteien verpflichten sich, etwaige
Nachzahlungen bzw. Guthaben innerhalb von vier Wochen nach Zugang der
Abrechnung an die jeweils andere Partei zu zahlen.

\subsubsection*{Abstandszahlung und Aufwendungsersatz}

Da die Aufhebung des Mietvertrags im Interesse der Untermieter*in erfolgt,
verpflichtet sich die Untermieter*in zu einer Abstandszahlung in Höhe von
\textbf{\CONTRACTbonus{} EUR}. Die Untermieter*in verpflichtet sich diesen
Betrag innerhalb von vier Wochen nach Beendigung des Mietverhältnisses auf
das im Abschnitt \emph{Räumung und Übergabe} genannte Konto zu überweisen.

\subsubsection*{Widerspruchsrecht des Mieters}

Die Mieter*in wird darauf hingewiesen, dass ihm kein Widerspruchsrecht
gemäß §574 BGB zusteht.

\subsubsection*{Sonstiges}

Soweit dieser Vertrag keine abweichenden Regelungen enthält, bleibt es bei
den mietvertraglichen Vereinbarungen.

Für den Fall, dass einzelne Bestimmungen dieses Vertrags unwirksam sind,
bleibt der Vertrag im Übrigen wirksam. An die Stelle der unwirksamen
Bestimmung tritt diejenige Vereinbarung, die dem Gewollten in rechtlich
zulässiger Weise wirtschaftlich am Besten entspricht.

% Ort und Datum  
\vspace{1,5 cm} 
\begin{tabular}{p{7cm}p{.5cm}l}
\dotfill \\ 
Ort, Datum
\end{tabular}%

% Hier kommen die Unterschriten hin
\vspace{1,00 cm} 
\begin{tabular}{p{7cm}p{.5cm}l}
\dotfill \\ 
Unterschrift Untermieter*in  \\
\MIETERfirstname~\MIETERlastname
\end{tabular}% 
\hfill 
\begin{tabular}{p{7cm}p{.5cm}l}
\dotfill \\ 
Unterschrift Hauptmieter*in \\
\HAUPTMIETERfirstname~\HAUPTMIETERlastname
\end{tabular}%
